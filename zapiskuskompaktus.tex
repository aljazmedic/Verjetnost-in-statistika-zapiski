\documentclass[a4paper,8pt]{extarticle}
\usepackage[utf8]{inputenc}

\usepackage{fancyhdr}

\usepackage[pdftex]{graphicx} % Required for including pictures
\usepackage[pdftex,linkcolor=black,pdfborder={0 0 0}]{hyperref} % Format links for pdf
\usepackage{calc} % To reset the counter in the document after title page
\usepackage{enumitem} % Includes lists

\usepackage{textcomp}
\usepackage{eurosym}

\usepackage{ dsfont } % font za množice
% tabele
\usepackage{array}
\usepackage{wrapfig}

\usepackage{tikz,forest}
\usetikzlibrary{arrows.meta}

\frenchspacing % No double spacing between sentences
\setlength{\parindent}{0pt}
\setlength{\parskip}{0.7em}

\usepackage{mathtools}
\usepackage{blkarray, bigstrut} %


\usepackage{amssymb,amsmath,amsthm,amsfonts}
\usepackage{multicol,multirow}
\usepackage{calc}
\usepackage{ifthen}
\usepackage{tabularx}
\usepackage[landscape]{geometry}
\usepackage{listings}
\usepackage{inconsolata}
%\usepackage[colorlinks=true,citecolor=blue,linkcolor=blue]{hyperref}
%\usepackage{accents}

\newcommand{\vect}[1]{\accentset{\rightharpoonup}{#1}}

\ifthenelse{\lengthtest { \paperwidth = 11in}}
    { \geometry{top=.5in,left=.5in,right=.5in,bottom=.5in} }
	{\ifthenelse{ \lengthtest{ \paperwidth = 297mm}}
		{\geometry{top=1cm,left=1cm,right=1cm,bottom=1cm} }
		{\geometry{top=1cm,left=1cm,right=1cm,bottom=1cm} }
	}
\pagestyle{empty}
\makeatletter
\renewcommand{\section}{\@startsection{section}{1}{0mm}%
                                {-1ex plus -.5ex minus -.2ex}%
                                {0.5ex plus .2ex}%x
                                {\normalfont\large\bfseries}}
\renewcommand{\subsection}{\@startsection{subsection}{2}{0mm}%
                                {-1explus -.5ex minus -.2ex}%
                                {0.5ex plus .2ex}%
                                {\normalfont\normalsize\bfseries}}
\renewcommand{\subsubsection}{\@startsection{subsubsection}{3}{0mm}%
                                {-1ex plus -.5ex minus -.2ex}%
                                {1ex plus .2ex}%
                                {\normalfont\small\bfseries}}
\makeatother
\setcounter{secnumdepth}{0}
%\setlength{\parindent}{0pt}
%\setlength{\parskip}{0pt plus 0.5ex}

% listings okolje za psevdo kodo
\lstnewenvironment{koda}[1][] %defines the algorithm listing environment
{   
    \lstset{ %this is the stype
        mathescape=true,
        basicstyle=\scriptsize, 
		columns=flexible,
        keywordstyle=\bfseries\em,
        keywords={,vhod, izhod, zacetek, konec, koncamo, ponavljaj, dokler, ce, vrni, za, vsak, vse, v, sicer,} %add the keywords you want, or load a language as Rubens explains in his comment above.
        xleftmargin=.1\textwidth,
		tabsize=4,
		%frame=leftline,xleftmargin=5pt,xrightmargin=5pt,framesep=5pt,
		%inputencoding = utf8,
		extendedchars = true,
		literate={ž}{{\ˇz}}1 {š}{{\ˇs}}1 {č}{{\ˇc}}1 {Ž}{{\ˇZ}}1 {Š}{{\ˇS}}1 {Č}{{\ˇC}}1,
        #1 % this is to add specific settings to an usage of this environment (for instnce, the caption and referable label)
    }
}
{}
% -----------------------------------------------------------------------

\begin{document} 

\begin{multicols}{5}
\setlength{\premulticols}{1pt}
\setlength{\postmulticols}{1pt}
\setlength{\multicolsep}{1pt}
\setlength{\columnsep}{2pt}

\section{Odvodi}
\setlength{\tabcolsep}{0.5em}{\renewcommand{\arraystretch}{1.2}
\begin{tabular}{ | r | l | }
    \hline
    \emph{funkcija} & \emph{odvod}\\\hline
    $c$ & $0$ \\ \hline
    $x^n$ & $nx^{n-1}$ \\ \hline
    $a^x$ & $a^x\ln{a}$ \\\hline
    $\frac{a^x}{\ln a}$ & $a^x$\\\hline
    $x^x$ & $x^x(1+\ln{x})$ \\\hline
    $\ln(x)$ & $\frac{1}{x}$ \\\hline
    $\log_{a}(x)$ & $\frac{1}{x\ln(a)}$ \\\hline
    $\sin(x)$ & $cos(x)$ \\\hline
    $\cos(x)$ & $-sin(x)$ \\\hline
    $\tan(x)$ & $\frac{1}{cos^2(x)}$ \\\hline
    $\cot(x)$ & $-\frac{1}{sin^2(x)}$ \\\hline
    $\arcsin(x)$ & $\frac{1}{\sqrt{1-x^2}}$ \\\hline
    $\arccos(x)$ & $-\frac{1}{\sqrt{1-x^2}}$ \\\hline
    $\arctan(x)$ & $\frac{1}{1+x^2}$ \\\hline
    $\textrm{arccot}(x)$ & $-\frac{1}{1+x^2}$ \\\hline
    $\textrm{sh}(x) = \frac{e^x - e^{-x}}{2}$ & $\textrm{ch}(x)$\\\hline
    $\textrm{ch}(x) = \frac{e^x + e^{-x}}{2}$ & $\textrm{sh}(x)$\\\hline
    $\textrm{th}(x) = \frac{\textrm{sh}(x)}{\textrm{ch}(x)}$ & $\frac{1}{\textrm{ch}^2(x)}$\\\hline
    $\textrm{cth}(x) = \frac{1}{\textrm{th}(x)}$ & $-\frac{1}{\textrm{sh}^2(x)}$\\\hline
    $\textrm{arsh}(x) = \ln(x+\sqrt{x^2+1})$ & $\frac{1}{\sqrt{1+x^2}}$\\\hline
    $\textrm{arch}(x) = \ln(x+\sqrt{x^2-1})$ & $\frac{1}{\sqrt{1-x^2}}$\\\hline
    $\textrm{arth}(x) = \frac{1}{2}\ln{\frac{1+x}{1-x}}$ & $\frac{1}{(1+x)(1-x)}$\\\hline
\end{tabular}
}

\section{Osnove kombinatorike}
\subsection*{Rodovne funkcije}
\[
	\begin{aligned}
		\sum_{n=0}^{\infty} q^n &= \frac{1}{1-q} &
		\sum_{n=0}^{b} q^n &= \frac{1-q^{b+1}}{1-q}
		\\
		\sum_{n=a}^{\infty} q^n &= \frac{q^{a}}{1-q} &
		\sum_{n=a}^{b} q^n &= \frac{q^a-q^{b+1}}{1-q}
	\end{aligned}
\]
\[
	a^n - b^n = (a-b)(a^{n-1} + a^{n-2}b + ... + ab^{n-2} + b^{n-1})  
\]
\[ \textstyle \frac{a_0 + ... + a_{k-1}x^{k-1}}{1-x^k} = a_0 + ... + a_{k-1}x^{k-1} + a_0^k + ... + a_{k-1}x^{2k-1} + ...\]
\[ (x+y)^n = \sum_{k=0}^{n} \binom{n}{k} x^{n-k}y^{k} \]
\[ \frac{1}{(1-x)^n} = \sum_{k=0}^{n} \binom{n+k-1}{k} x^{k} \]
\[ B_\lambda(x) = \sum_{n} \binom{\lambda}{n} x^{n} = (1+x)^\lambda; \qquad \binom{\lambda}{n} = \frac{\lambda^{\underline{n}}}{n!}\]

\subsection*{Izbori}
Imamo $n$ oštevilčenih kroglic. Na koliko načinov lahko izberemo $k$ kroglic?

\begin{center}
    \begin{tabular}{ m{6em} | c | c | } 
         & \textbf{s pon.} & \textbf{brez pon.}\\ 
        \hline
        \textbf{variacije} \emph{vrstni red je pomemben} & $n^k$ & $n^{\underline{k}}$ \\ 
        \hline
        \textbf{kombinacije} \emph{vrstni red ni pomemben} & $\binom{n+k-1}{k}$ & $\binom{n}{k}$ \\ 
    \end{tabular}
\end{center}

\[\binom{n}{k} = \frac{n^{\underline{k}}}{k!} = \frac{n!}{k!(n-k)!} = \binom{n}{n-k}\]



\subsubsection{Pravila za računanje z dogodki}
\begin{align*}
	\text{idempotentnost} & \quad A \cup A = A = A \cap A \\
	\text{komutativnost} & \quad A \cup B = B \cup A \\
	& \quad A \cap B = B \cap A \\
	\text{asociativnost} & \quad (A \cup B) \cup C = A \cup ( B \cup C) \\
	& \quad (A \cap B) \cap C = A \cap ( B \cap C) \\
	\text{distibutivnost} & \quad (A \cup B) \cap C = (A \cap C) \cup ( A \cap C) \\
	& \quad (A \cap B) \cup C = (A \cap C) \cup ( A \cap C) \\
	\text{De Morgan} & \quad \big(\bigcup_{i\in I} A_i \big)^\complement = \bigcap_{i \in I} A_i^\complement  \\
	& \quad \big(\bigcap_{i\in I} A_i \big)^\complement = \bigcup_{i \in I} A_i^\complement \\
	& \quad A \cup \Omega = \Omega \quad  A \cap \Omega = A \\
	& \quad A \cup \emptyset = A \quad  A \cap \emptyset = \emptyset \\
	& \quad A \cup A^\complement = \Omega \quad A \cap A^\complement = \emptyset
\end{align*}


Neprazna družina dogodkov $\mathcal{F}$ v $\Omega$ je $\sigma$-algebra, če velja
\begin{itemize}
	\item zaprtost za komplemente: \[ A \in \mathcal{F} \implies A^\complement \in \mathcal{F} \]
	\item zaprtost za števne unije: \[ A_1, A_2, \dots \in \mathcal{F} \implies \bigcup_{i=1}^\infty A_i \in \mathcal{F} \]
\end{itemize}
\textit{Če zahtevamo zaprtost le za končne unije, je $\mathcal{F}$ le algebra.}

Ker je po De Morganovem zakonu $\big(\bigcup_{i\in I} A_i^\complement \big)^\complement = \bigcap_{i \in I} A_i$ imamo zaprtost tudi za preseke.

Ker je $A \setminus B = A \cap B^\complement$ je algebra zaprta tudi za razlike.

Najmanjša algebra je \textbf{trivialna}: $ \{ \emptyset, \Omega \}$.

Največja algebra je: $\mathcal{P}(\Omega)$.

Najmanjša algebra, ki vsebuje $E$ je $\{ \emptyset, E, E^\complement, \Omega \}$.

Dogodka $A$ in $B$ sta \textbf{nezdružljiva} (disjunktna), če je $A \cup B = \emptyset$.

Zaporedje $\{ A_i \}_i \in \mathcal{F}$ (končno ali števno mnogo) je \textbf{popoln sistem dogodkov}, če
\begin{align*}
	\bigcup_i A_i &= \Omega & A_i \cup A_j &= \emptyset, \, \forall i,j: i\neq j
\end{align*}


\textbf{Verjetnost} na $(\Omega, \mathcal{F})$ je preslikava $P: \mathcal{F} \to \mathbb{R}$ z lastnostmi:

\begin{itemize}
	\item $P(A) \geq 0$ za $\forall A \in \mathcal{F}$
	\item $P(\Omega) = 1$
	\item Za paroma nezdružljive dogodke $\{ A_i \}_{i=1}^\infty $ velja \textit{števna aditivnost}
	\[ P(\bigcup_{i=1}^\infty A_i) = \sum_{i=1}^\infty P(A_i)\]
\end{itemize}

Lastnosti $P$:
\begin{itemize}
	\item $P(\emptyset) = 0$
	\item $P$ je končno aditivna.
	\item $P$ je \textit{monotona}: $A \subseteq B \implies P(A) \leq P(B)$
	\item $P(A^\complement) = 1 - P(A)$
	\item $P$ je \textit{zvezna}:
	\[ A_1 \subseteq A_2 \subseteq \dots \implies P\big(\bigcup_{i=1}^\infty\big) = \lim_{i \to \infty} P(A_i)\]
	\[ B_1 \supseteq B_2 \supseteq \dots \implies P\big(\bigcap_{i=1}^\infty\big) = \lim_{i \to \infty} P(B_i)\]
\end{itemize}


\[P(A^\complement) = 1 - P(A)\]
\[ P(A \cup B) = P(A) + P(B) - P(A \cap B) \]

Če lahko prostor izidov razbijemo na paroma nezdružljive enako verjetne dogodke, jih lahko obravnavamo kot izide: če je dogodek $A$ unija $k$ od $n$ takih dogodkov, je $P(A) = k/n$.

\subsection{Načelo vključitev in izključitev}
\begin{align*}
	P(A_1 \cup A_2) &= P(A_1) + P(A_2) - P(A_1 \cap A_2)
\end{align*}
\[ P( \bigcup_{i=1}^n A_i) = \sum_{\emptyset \neq S \subseteq [n]} (-1)^{|S|+1} P(\bigcap_{i\in S} A_i)\]

\subsection{Pogojna verjetnost}
\[ P(A | B) = \frac{P(A \cap B)}{P(B)}\]


\subsection{Izrek o polni verjetnosti}
Če $H_1, H_2, H_3, \dots $ tvorijo \textbf{popoln sistem dogodkov} \textit{(tj. vedno se zgodi natanko eden izmed njih)}, velja:
\[ P(A) = P(H_1) P(A | H_1) + P(H_2) P(A | H_2) + \dots \]
Dogodkom $H_i$ često pravimo hipoteze in jih je lahko končno ali pa števno
neskončno

\subsection{Bayesova formula}
\[ P(H_i | A) = \frac{P(H_i)P(A | H_i)}{P(A)}\]
\[ P(H_i | A) = \frac{P(H_i) P(A | H_i)}{P(H1) P(A | H1) + P(H2) P(A | H2) + \dots } \]

Brezpogojnim verjetnostim $P(H_i)$ pravimo \textbf{apriorne}, pogojnim verjetnostim
$P(H_i | A)$ pa \textbf{aposteriorne} verjetnosti hipotez.

\subsection{Neodvisnot dogodkov}

Dogodka A in B sta \textbf{neodvisna}, če velja:
\[ P(A \cap B) = P(A) P(B) \]
Če je $P(B) > 0$, je to ekvivalentno pogoju, da je $P(A | B) = P(A)$.
Če je $0 < P(B) < 1$, pa je to ekvivalentno tudi pogoju, da je $P(A | B) = P(A | B^\complement)$.
Dogodki $A1, A2, A3, \dots$ so neodvisni, če za poljubne različne indekse $i_1, i_2, \dots , i_k$ velja:
\[ P(A_{i_1} \cap A_{i_2} \cap \dots \cap A_{i_k}) = P(A_{i_1}) P(A_{i_2}) \dots P(A_{i_k}) \]

\section{Slučajne spremenljivke}
Slučajna spremenljivka je funkcija $X: \Omega \to \mathbb{R}$ z lastnostijo, da je $\forall x \in \mathbb{R}$ množica $\{ \omega \in \Omega : X(\omega) < x \} \equiv X^{-1}((-\infty, x]) \equiv (X \leq x)$ dogodek.

\subsubsection{Diskretne porazdelitve}
\textbf{Diskretna enakomerna porazdelitev} na $n$ točkah:
\[ X \sim \binom{a_1\ a_2\ \dots\ a_n}{\frac{1}{n}\ \frac{1}{n}\ \dots\ \frac{1}{n}} = \text{Unif} \{ a_1 , \dots , a_n \} \]

\textbf{Binomska porazdelitev} \\
$\text{Bin}(n, p)$, $n \in \mathbb{N}$, $p \in (0,1)$

Naj bo $X$ št. uspelih (z verjetnostjo $p$) poskusov v zaporedju $n$ poskusov. $X \sim \text{Bin}(n, p)$:
\[ p_k = P(X = k) = \binom{n}{k} p^k (1-p)^{n-k}\]

\textbf{Bernulijeva porazdelitev} $Ber(p) \sim Bin(1, p)$

\textbf{Geometrijske porazdelitev} \\
$\text{Geo}(p)$,  $p \in (0,1)$

$(X = k)$ je dogodek, da se $A$ zgodi prvič v $k$-ti ponovitvi.

\[P(X = k) = (1-p)^{k-1} p\]

\textbf{Pascalova / negativna binomska porazdelitev} \\
$Pas(m, p) = NB(m, p)$, $m \in \mathbb{N}$, $p \in (0,1)$

$(X = k)$ je dogodek, da se dogodek $A$ zgodi $m$-tič v $k$-ti ponovitvi.

Oziroma $X$ je število poskusov do vključno $m$-tega uspelega, pri katerig vsak uspe z verjetnostjo $p$. $X \sim Pas(m, p)$:
\[ p_k = P(X = k) = \binom{k-1}{n-1} p^k (1-p)^{k-n}\]

\textbf{Hipergeometrijska porazdelitev} \\
Iz posode, v kateri je $n$ kroglic, od tega $r$ rdečih, na slepo in brez vračanja
izvlečemo $s$ kroglic. Če z $X$ označimo število rdečih med izvlečenimi, ima
ta slučajna spremenljivka hipergeometrijsko porazdelitev: $X \sim \text{Hip}(s, r, n) = \text{Hip}(r, s, n)$. Velja

\[ P(X = k) = \frac{\binom{r}{k} \binom{n-r}{s-k}}{\binom{n}{s}} = \frac{\binom{s}{k} \binom{n-s}{r-k}}{\binom{n}{r}}\]


\subsubsection{Aproksimacija binomske porazdelitve}
\textbf{Poissonova porazdelitev} \\
$Poi(\lambda)$, $\lambda > 0$

\[ p_k = P(X = k) = \frac{\lambda^k}{k!} e^{-\lambda}\]

Če imamo veliko ponovitev ($n \to \infty$) z malo verjetnostjo ($p \to 0$), je $\text{Bin}(n, p) \approx \text{Poi}(np)$

\textbf{Laplaceova lokalna formula: }
Če je $p, 1-p \gg \frac{1}{n}$, lahko $X \sim \text{Bin}(n, p)$ aproksimiramo
\[ P(X = k) \approx \frac{1}{\sigma\sqrt{2\pi}} e^{-\frac{(k-np)^2}{2\sigma^2}} \quad \sigma = \sqrt{np(1-p)} \]

\textbf{Laplaceova integralska formula: }
\[ P(a \leq X \leq b) \approx \Phi\left( \frac{b - np}{\sigma} \right) - \Phi\left( \frac{a - np}{\sigma} \right) \]

Za majhno relativno napako zahtevamo še: 
\begin{itemize}
	\item $|a-np| \ll \sigma^{4/3}$ ali $|b-np| \ll \sigma^{4/3}$ 
	\item $a, b \in \mathbb{Z} + \frac{1}{2}$ ali $b - a \gg 1$
\end{itemize}

\subsubsection{Kumulativna porazdelitvena funkcija}
\[ F_X(x) = P(X \leq x)\]

\subsection{Zvezno porazdeljene slučajne spremenljivke}
Realna slučajna spremenljivka $X$ je porazdeljena zvezno, če obstaja taka integrabilna funkcija $p_X: \mathbb{R} \to [0, \infty)$, da za poljubna $a \leq b$ velja:
\[ P( a \leq X \leq b) = \int_a^b p_X(t) dx \]
Funkciji $p_X$ pravimo \textbf{porazdelitvena gostota}


Komulativna funkcija slučajne spremenljivke $X$
\[ F_X(x) = \int_{\infty}^x p_X(t) dt \]

Če je $F_X$ zvezna in odsekoma zvezno odvedljiva, je $X$ porazdeljena zvezno in za vse razen za končno mnogo točk $x$ velja $p_x(x) = F'_X(x)$.

\subsection{Zvezne porazdelitve}
\textbf{Enakomerna zvezna porazdelitev} na $ [a, b] $

\[ p(x) = \begin{cases}
	\frac{1}{b-a} & a \leq x \leq b \\
	0 & \text{sicer}
\end{cases}\]

\[ F(x) = \begin{cases}
	0 & x \leq a \\
	\frac{x-a}{b-a} & a \leq x \leq b \\
	1 & b \leq x
\end{cases}\]

\textbf{Normalna / Gaussova porazdelitev} $N(\mu, \sigma)$
\[ p(x) = \frac{1}{\sigma \sqrt{2\pi}} e^{-\frac{1}{2} \left( \frac{x-\mu}{\sigma} \right)^2}\]

\[ F(x) = \frac{1}{2} + \Phi\left(\frac{x-\mu}{\sigma}\right)\]

Standardizirana normalna porazdelitev: $N(0, 1)$

\textbf{Eksponentna porazdelitev} $\text{Exp}(\lambda)$
\[ p(x) = \begin{cases}
	\lambda e^{-\lambda x} & x \geq 0 \\
	0 & x < 0
\end{cases}\]

\[ F(x) = \begin{cases}
	1-e^{-\lambda x} & x \geq 0 \\
	0 & x < 0
\end{cases}\]


\textbf{Porazdelitev} $\Gamma(b,c)$, $b > 0$, $c > 0$
\[ p(x) = \begin{cases}
	\frac{c^b}{\Gamma(b)} x^{b-1} e^{-cx} & x \geq 0 \\
	0 & x < 0
\end{cases}\]

Funkcija $\Gamma$
\begin{gather*}
	\Gamma(s) = \int_0^{\infty} x^{s-1} e^{-x} dx, \qquad \forall s > 0 \\
	\Gamma(1) = 1 \qquad \Gamma(s+1) = s \Gamma(s) \qquad \Gamma(\frac{1}{2}) = \sqrt{\pi} \\
	\Gamma(n+1) = n!\qquad n\in \mathbb{N} \\ 
	\Gamma(x)\Gamma(x+1) = \frac{\pi}{\sin(\pi x)} 
\end{gather*}

\textbf{Porazdelitev} $\chi^2(n)$ \\
$n \in \mathbb{N}$ je št. prostorskih stopenj

\[ \chi^2(n) = \Gamma(\frac{n}{2}, \frac{1}{2})\]

\textbf{Cauchyjeva porazdelitev}
\[ p(x) = \frac{1}{\pi (1+x^2)}\]
\[ F(x) = \frac{1}{\pi} \arctan x + \frac{1}{2}\]

\section{Slučajni vektorji}
Slučajni vektor je $n$-terica slučajnih spremenljivk $X = (X_1, \dots , X_n): \Omega -> R$ 

\subsubsection{Porazdelitvena funkcija slučajnega vektorja}
\[ F_{(X_1, \dots , X_n)}(x_1, \dots , x_n) = P(X_1 \leq x_1, \dots , X_n \leq x_n)\]

\subsection{Neodvisnot slučajnih spremenljivk}
Slučajne spremenljivke so neodvisne, če je
\[ F_{(X_1, \dots , X_n)}(x_1, \dots, x_n) = F_{X_1}(x_1) \dots F_{X_n}(x_n)\]
Torej so dogodki $(X_1 \leq x_1), \dots , (X_n \leq x_n)$ neodvisni.

Naj bo $(X, Y)$ diskretno porazdeljen sluč. vektor:
\[ p_{ij} = P(X = x_i, Y = y_j) \quad p_i = P(X = x_i) \quad P(Y = y_j) \]
potem velja:
\[ X, Y \text{neodvisni} \iff p_{ij} = p_i q_j \]


Naj bo $(X, Y)$ zvezno porazdeljen sluč. vektor z gostoto $ p_{(X, Y)}(x,y)$, potem velja:
\[ X, Y \text{neodvisni} \iff \exists p_X, p_Y: p_{(X,Y)}(x,y) = p_X(x) p_Y(y)\]

\subsection{Funkcije slučajnih spremenljivk}
Naj bosta $A, B \subseteq \mathbb{R}$ odprti množici in $h: A \to B$ taka bijekcija, da je funkcija
$h^{-1}: B \to A$ zvezno odvedljiva. Nadalje naj bo $X$ zvezno porazdeljena slučajna
spremenljivka z gostoto $p_X$, ki je izven množice A enaka nič. Tedaj je slučajna
spremenljivka $Y := h(X)$ porazdeljena zvezno z gostoto:

\[ p_Y(y) = \begin{cases}
	p_X\left( h^{-1}(y) \right) \left|(h^{-1})'(y)\right| & y \in B \\
	0 & sicer	
\end{cases}\]

Naj bo $X$ zvezno porazdeljena slučajna spremenljivka z gostoto $p_X$, skoncentrirana na odprti množici $A$. Naj bo $h: A \to \mathbb{R}$ zvezno odvedljiva in $h'(x) \neq 0$ za $\forall x \in A$. 
Tedaj je slučajna spremenljivka $Y = h(X)$ porazdeljena zvezno z gostoto:

\[ p_Y = \sum_{\substack{x \in A \\ h(x) = y}} \frac{p_X(x)}{|h'(x)|}\]


\subsection{Matematično upanje}
Diskretno porazdeljena sl. sprem. \\
$ X \sim \left( \begin{matrix}
	x_1 & x_2 & \dots \\
	p_1 & p_2 & \dots
\end{matrix}\right) $

\[ E(X) = \sum_{k=1}^\infty x_k p_k \quad \text{če} \quad \sum_{k=1}^\infty |x_k| p_k < \infty \]

Zvezno porazdeljena sl. sprem. $X$ z gostoto $p_X$
\[E(X) = \int_{-\infty}^{\infty} x p_X(x) dx \quad \text{če} \quad \int_{-\infty}^{\infty} |x| p_X(x) dx  < \infty\]

\subsubsection{Lastnosti}
Naj bo $f: \mathbb{R} \to \mathbb{R}$ zvezna funkcija. Potem je
\[ E(f(X)) = \sum_{k=1}^\infty f(x_k) p_k \quad \text{če obstaja}\]
\[ E(f(X)) = \int_{-\infty}^\infty f(x) p_X(x) dx \quad \text{če obstaja}\]

Če ima $|X|$ mat. up., ga ima tudi $X$ in velja 
\[|E(X)| \leq E(|X|) \]

Če obstaja $E(X^2)$ in $E(Y^2)$, obstaja tudi $E(XY)$ in velja:
\[|E(XY)| \leq E(|XY|) \leq \sqrt{E(X^2)E(Y^2)} \]

Za poljubne sl. sprem $X_1, \dots, X_n$ velja:
\[ E(a_1 X_1 + \dots a_n X_n) = a_1 E(X_1) + \dots + a_n E(X_n) \]

\subsection{Disperzija (varianca)}
\[D(X) = E((X - E(X))^2) = E(X^2) - (E(X))^2\]
Lastnosti: 
\begin{itemize}
	\item $D(X) \geq 0$
	\item $D(X) = 0 \iff P(X = E(X)) = 1$
	\item $D(aX) = a^2 D(X)$
\end{itemize}

Standardna diviacija/odklon:
\[ \sigma(X) = \sqrt{D(X)} \]
zanjo velja $\sigma (aX) = |a|\sigma(X)$.


\subsection{Nekoreliranost}
Sl. sprem. $X$ in $Y$ sta nekorelirani, če velja:
\[ E(XY) = E(X)E(Y) \]
\[ X, Y \text{ neodvisni } \implies X,Y \text{ nekorelirani }\]

Če imata $X$ in $Y$, je nekoreliranost ekvivalentna zvezi:
\[ D(X+Y) = D(X) + D(Y)\]

\subsection{Kovarianca}
\begin{align*}
	K(X,Y) &= E((X-E(X))(Y-E(Y))) \\
	&= E(XY)-E(X)E(Y)
\end{align*}

\begin{itemize}
	\item $K(X,X) = D(X)$
	\item $K(X,Y) = 0 \iff X,Y \text{nekorelirani}$
	\item $K(aX,bY, Z) = aK(X, Z) + bK(Y, Z)$
	\item $K(X,Y) = K(Y,X)$
	\item $K(aX+b,cY+d) = acK(X,Y)$
	\item $|K(X,Y)| \leq \sqrt{D(X)D(Y)}$
	\item $D(X+Y) = D(X) + D(Y) + 2K(X,Y)$ 
	\item $D(X_1 + \dots + X_n) = D(X_1) + \dots + D(X_n) + 2\sum_{i=1}^{n-1}\sum_{j=i+1}^{n} K(X_i, X_j)$
\end{itemize}

\subsection{Standardizacija}
\[X_S = \frac{X-E(X)}{\sigma(X)}\]

\subsection{Korelacijski koeficient}
\[r(X,Y) = \frac{K(X, Y)}{ \sigma(X) \sigma(Y)} = E(X_S, Y_S)\]

Lastnosti:
\begin{itemize}
	\item $r(X,Y) = 0 \iff X, Y \text{nekorelirani}$
	\item $-1 \leq r(X,Y) \leq 1$
	\item $r(X,Y) = 1 \iff P(X_S = Y_S) = 1$
	\item $r(X,Y) = -1 \iff P(X_S = -Y_S) = 1$
	\item $r(aX+b, cY+d) = r(X, Y)$
\end{itemize}

\section{Pogojne porazdelitve}
Pogojna porazdelitev sl. sprem. $X$ glede na dogodek $B$:
\[ X|B \sim \left( \begin{matrix}
	a_1 & a_2 &\dots \\
	P(X = a_1 | B) & P(X = a_2 | B) &\dots 
\end{matrix}\right)\]

\subsection{Pogojna porazdelitvena funkcija}
sl. sprem. $X$ glede na dogodek $B$:
\[ F_{X|B}(x) = F_X(x|B) = P(X \leq x | B) = \frac{P((X\leq x) \cap B)}{P(B)} \]

Če je pogojna porazdelitev zvezna, objstaja tudi \textbf{pogojna porazdelitvena gostota}:
\[ p_{X|B}(x) = F'_{X|B}(x)\]

\subsection{Pogojno matematično upanje}
\[ E(h(X) | B) = \sum_x h(x) P(X = x | B)\]

Za vsako slučajno spremenlivko $X$ in dogodek $B$ veleja:
\[ E(X |B) = \frac{E(XZ)}{P(B)} = \frac{E(XZ)}{E(Z)}\]
kjer je sl. sprem. $Z$ indikator dogodka $B$.

Za vsako sl. sprem. $X$ z mat. up. in popoln sistem dogodkov $H_1, H_2, \dots$ velja \textbf{izrek o polni pričakovani vrednosti}
\[E(X) = P(H_1)E(X | H_1) + P(H_2)E(X | H_2)+\dots \]

\subsection{Regresijska funkcija}
\[ \varphi(y) = E(X | Y = y) \]

Za vsako sl. sprem. $X$ z mat. up. in diskretno sl. sprem. $Y$ velja:
\[ E(Xg(Y) | Y) = E(X | Y)g(Y) \]
\[ E(Xg(Y)) = E(E(X | Y)g(Y)) \]
\[ E(X) = E(E(X|Y))\]

Za vsak dododek $A$ in vsako sl. sprem $Y$ velja:
\[ E(P(A|Y)) = P(A)\]


\section{Momenti}
Moment reda $k$ glede na točko $a$ je
\[ m_k(a) = E((X-a)^k) \quad \textit{če obstaja}\]

\begin{itemize}
	\item \textbf{Začetni moment} $z_k := m_k(0) = E(X^k)$
	\item \textbf{Centralni moment} $m_k := m_k(E(X)) = E((X-E(x))^k)$
\end{itemize}

\[ z_1 = E(X) \qquad m_2 = D(X)\]

Če obstaja $m_n(a)$, obstaja tudi $m_k(a)$ za $\forall k < n$.

Če obstaja $z_n$, obstaja tudi $m_n(a)$ za $\forall a \in \mathbb{R}$

Centralne momente lahko izračunamo iz začetnih:
\[ m_n = \sum_{k=0}^n \binom{n}{k} (-1)^{n-k} z_1^{n-k} z_k \]

\subsection{Asimetrija}
\[A(X) = E(X_S^3) = E\left( (\frac{X-E(X)}{\sigma(X)})^3 \right) = \frac{m_3}{m_2^{\frac{3}{2}}}\]

$\forall \lambda > 0: A(\lambda X) = A(X)$

\subsection{Sploščenost (kurtozis)}
\[ K(X) = E\left[ \left( \frac{X-E(X)}{\sqrt{D(X)}}\right)^4\right] = \frac{m_4}{m_2^2}\]

Presežna sploščenost:
\[ K^*(X) = K(X) - 3\]

\section{Vrstilne karakteristike}

\subsubsection{Kvantil reda $p$}
je vsaka vrednost $x_p$, za katero velja:
\[ P(X \leq x_p) \geq p \text{ in } P(X \geq x_p) = 1 - p \]
oz. $F(x_p-) \leq p \leq F(x_p)$

\begin{itemize}
	\item Mediana: $x_{\frac{1}{2}}$
	\item Kvartili: $x_{\frac{1}{4}}, x_{\frac{2}{4}}, x_{\frac{3}{4}}$
	\item (Per)centili: $x_{\frac{1}{100}}, \dots, x_{\frac{99}{100}}$
\end{itemize}

\subsection{Semi interkvartilni razmik}
\[ s = \frac{1}{2}\left(x_{\frac{3}{4}} - x_{\frac{1}{4}} \right)\]

\section{Rodovne funkcije}
Naj bo $X$ sl. sprem. z zalogo vrednosti $\mathbb{N} \cup \{0\}$: 
\[ p_k = P(X=k) \qquad k = 0,1,2,\dots \]

Rodovna funkcija sl. sprem. $X$:
\[G_X(s) = p_0 + p_1 s + p_2 s^2 + \dots = \sum_{k=0}^\infty p_k s^k \]
Obstaja za vse $|s| \leq 1$.

\[ P(X=k) = \frac{G_X^{(k)(0)}}{k!} \]

\[ G_X(0) = p_0 \quad G_X(1) = 1 \quad G_X(s) = E(s^X) \]

Izrek o enoličnosti:
\begin{multline*}
\forall s \in [-1, 1]: G_X(s) = G_Y(s) \\
\iff P(X=k) = P(Y=k)\ \forall k = 0, 1, 2, \dots
\end{multline*}

\[ \lim_{s \uparrow 1} G'_X(s) = \lim_{s \uparrow 1} \sum_{k=1}^\infty k p_k s^{k-1} =  \sum_{k=1}^\infty  \lim_{s \uparrow 1} k p_k s^{k-1}  E(X)\]

Naj bo $X$ sl. sprem. z rodovno funkcijo $G_X$, potem je:
\[ G_X^{(n)}(1-) = E(X)(X-1)(X-2)\dots (X-n+1))\]

Naj bosta $X_1, \dots , X_n$ nedovisne sl. sprem. z rodovnimi funkcijami $G_{X_1}, \dots G_{X_n}$:
\[ G_{X_1+\dots + X_n} = G(X_1) \dots G(X_n) \qquad \forall s \in [-1, 1]\]


Naj bodo $\forall n \in \mathbb{N}$ sl. sprem $N, X_1, \dots, X_n$ neodvisne. Naj ima $N$ rodovno funkcijo $G_N$ in $X_i$ rodovno funkcijo $G_X$ ($X_1, \dots, X_n$ so enako porazdeljene).
Naj bo $S = X_1 +  \dots + X_n$. Potem je:
\[ G_S = G_N(G_X(s)) \qquad \forall s \in [-1, 1]\]

Velja tudi $E(S) = E(N)E(X)$.

\section{Momentno rodovna funkcija}
\begin{align*}
	M_X(t) &= E(e^{tX}) \qquad \forall t \in R \quad \textit{če obstaja} \\
	&= 1 + z_1t + \frac{z_2}{2!} t^2 + \frac{z_3}{3!} t^3 + \dots 
\end{align*} 

V primeru, ko ima $X$ zalogo vrednosti v $\mathbb{N} \cup \{0\}$, je 
\[M_X(t) = E(e^{tX}) = G_X(e^t) \]

Za zvezno porazdeljeno sl. sprem. $X$ velja:
\[ M_X(t) = \int_{-\infty}^\infty e^{tx} p_X(x) dx \]


Naj pri nekem $\delta > 0$ $M_X(t)$ obstaja za vse $t \in (-\delta, \delta)$.
Potem je porazdelitev za $X$ natanko določana z $M_X$ in vsi začetni momenti obstajajo:
\[ z_k = E(X^k) = M_X^{(k)}(0) \qquad \forall k \in \mathbb{N}\]
\[ M_X(t) = \sum_{k=0}^\infty \frac{z_k}{k!}t^k \qquad \forall t \in (-\delta, \delta)\] 

Trditev:
\[M_{aX+b}(t) = e^{bt} M_X(at) \]

Če sta $X$ in $Y$ neodvisni, je:
\[ M_{X+Y} (t) = M_X(t) M_Y(t)\]

\section{Izreka o velikih številih}

Zaporedje sl. sprem. $\left\{X_n\right\}_{n\in \mathbb{N}}$ \textbf{verjetnostno} konvergira proti sl. sprem. $X$, če
\[\forall \varepsilon > 0: \lim_{n \to \infty} P(|X_n - X| \geq \varepsilon) = 0\]
\[\forall \varepsilon > 0: \lim_{n \to \infty} P(|X_n - X| \leq \varepsilon) = 1\]
Zaporedje sl. sprem. $\left\{X_n\right\}_{n\in \mathbb{N}}$ \textbf{skoraj gotovo} (s.g.) konvergira proti sl. sprem. $X$, če
\[ P(\lim_{n \to \infty}) = 1 \]

Če $X_n \xrightarrow[n \to \infty]{s.g.} X$, potem
\[\forall \varepsilon > 0: \lim_{m \to \infty} P(|X_n - X| < \varepsilon, \forall n \geq m) = 1\]

Če $X_n \xrightarrow[n \to \infty]{s.g.} X$, potem $X_n \xrightarrow[n \to \infty]{ver.} X$


Naj bo $X_1, X_2, X_n, \dots$ zaporedje sl. sprem. z mat. up. in naj bo $S = X_1 +\dots + X_n$, $Y_n = \frac{S_n - E(S_n)}{n}$. Potem je $E(Y_n) = 0$.

Za $\left\{ X_n \right\}_{n \in \mathbb{N}} $ velja \textbf{šibki zakon o velikih številih}, če zap. $\left\{ Y_n \right\}_{n \in \mathbb{N}}$
konvergira proti 0 verjetnostno, tj.
\[ \forall \varepsilon > 0: \lim_{n \to \infty} P\left(  \frac{|S_n - E(S_n)|}{n} < \varepsilon \right) = 1 \]

Za $\left\{ X_n \right\}_{n \in \mathbb{N}} $ velja \textbf{krepki zakon o velikih številih}, če zap. $\left\{ Y_n \right\}_{n \in \mathbb{N}}$
konvergira proti 0 skoraj gotovo, tj.
\[ P\left( \lim_{n \to \infty} \frac{S_n - E(S_n)}{n} = 0 \right) = 1 \]

\subsection{Neenakost Markova}
Če je $X$ sl. sprem. z mat. up., potem je
\[P(|X| \geq a) \leq \frac{E(|X|)}{a} \quad \forall a > 0\]

\subsection{Neenakost Čebiševa}
Če ima $X$ disperzijo, je
\[ P(|X - E(X)| \geq a \sigma(X)) \leq \frac{1}{a^2}  \quad \forall a > 0\]
oziroma za $\varepsilon := a \sigma(X)$
\[ P(|X-E(X)| \geq \varepsilon) \leq \frac{D(X)}{\varepsilon^2}\]

\textit{Izrek Markova:}
Če za sl. sprem. $\left\{ X_n \right\}_{n \in \mathbb{N}} $ velja $\frac{D(S_n)}{n^2} \xrightarrow[n \to \infty]{} 0$, potem velja ŠZVŠ.

\textit{Izrek Čibiševa:}
Če so sl. sprem. $ X_1, X_2, \dots $ paroma nekorelirane in je $\sup_{n \in \mathbb{N}} D(X_n) < \infty$, potem velja ŠZVŠ.

\textit{Izrek Kolmogorova}
Naj za neodvisne slučajne spremenljivke $X_1, X_2, \dots$ velja pogoj
\[ \sum_{n=1}^\infty \frac{D(X_n)}{n^2} < \infty\]
potem za  $\left\{ X_n \right\}_{n \in \mathbb{N}} $ velja KZVŠ.

Zgornji pogoj velja, če je $\sup_n D(X_n) < \infty$

Naj bodo  $X_1, X_2, \dots$ neodvisne in enako porazdeljene sl. sprem. z disperzijo. Potem velja KZVŠ.

\section{Centralni limitni izrek}
Naj bo $X_1, X_2, X_n, \dots$ zaporedje sl. sprem. z mat. up. in naj bo $S = X_1 +\dots + X_n$, $Z_n = \frac{S_n - E(S_n)}{n}$. Potem je $E(Z_n) = 0$ in $D(Z_n) = 1$.

Za $\left\{ X_n \right\}_{n \in \mathbb{N}} $ veljaj \textbf{centralni limitni zakon}, če
\[ F_{Z_n}(x) \xrightarrow[n \to \infty]{} F_{N(0,1)}  \quad \forall x \in \mathbb{R}\]

Če so $X_1, X_2, \dots$ neodvisne in enako porazdeljene, velja centralni limitni zakon.
\textit{Za velike $n$ je $S_n \sim N(E(S_n), \sigma(S_n))$}

\[ P\left(\frac{S_n - E(S_n)}{\sigma(S_n)} \leq x \right) \xrightarrow[n \to \infty]{} \frac{1}{\sqrt{2\pi}} \int_{-\infty}^{x} e^{\frac{t^2}{2}} dt\]
\[ P(a \leq S_n \leq b) \approx \Phi\left(\frac{b-E(S_n)}{\sigma(S_n)}\right) - \Phi\left(\frac{a-E(S_n)}{\sigma(S_n)}\right)  \]


\subsubsection{Izrek o zveznosti rodovne funkcije}

Naj za zaporedje $\left\{ Z_n \right\}_{n \in \mathbb{N}} $ sl. sprem. velja
\[ M_{Z_n}(t) \to M_{N(0,1)}(t) = e^{\frac{t^2}{2}} \qquad \forall t \in (-\delta, \delta)\]
potem
\[ F_{Z_n}(x) \to F_{N(0,1)}(x)  \qquad \forall x \in \mathbb{R}\]
\end{multicols}
\end{document}