\documentclass[a4paper,8pt]{extarticle}
\usepackage[utf8]{inputenc}

\usepackage{fancyhdr}

\usepackage[pdftex]{graphicx} % Required for including pictures
\usepackage[pdftex,linkcolor=black,pdfborder={0 0 0}]{hyperref} % Format links for pdf
\usepackage{calc} % To reset the counter in the document after title page
\usepackage{enumitem} % Includes lists

\usepackage{textcomp}
\usepackage{eurosym}

\usepackage{ dsfont } % font za množice
% tabele
\usepackage{array}
\usepackage{wrapfig}

\usepackage{tikz,forest}
\usetikzlibrary{arrows.meta}

\frenchspacing % No double spacing between sentences
\setlength{\parindent}{0pt}
\setlength{\parskip}{0.7em}

\usepackage{mathtools}
\usepackage{blkarray, bigstrut} %


\usepackage{amssymb,amsmath,amsthm,amsfonts}
\usepackage{multicol,multirow}
\usepackage{calc}
\usepackage{ifthen}
\usepackage{tabularx}
\usepackage[landscape]{geometry}
\usepackage{listings}
\usepackage{inconsolata}
%\usepackage[colorlinks=true,citecolor=blue,linkcolor=blue]{hyperref}
%\usepackage{accents}

\newcommand{\vect}[1]{\accentset{\rightharpoonup}{#1}}

\ifthenelse{\lengthtest { \paperwidth = 11in}}
    { \geometry{top=.5in,left=.5in,right=.5in,bottom=.5in} }
	{\ifthenelse{ \lengthtest{ \paperwidth = 297mm}}
		{\geometry{top=1cm,left=1cm,right=1cm,bottom=1cm} }
		{\geometry{top=1cm,left=1cm,right=1cm,bottom=1cm} }
	}
\pagestyle{empty}
\makeatletter
\renewcommand{\section}{\@startsection{section}{1}{0mm}%
                                {-1ex plus -.5ex minus -.2ex}%
                                {0.5ex plus .2ex}%x
                                {\normalfont\large\bfseries}}
\renewcommand{\subsection}{\@startsection{subsection}{2}{0mm}%
                                {-1explus -.5ex minus -.2ex}%
                                {0.5ex plus .2ex}%
                                {\normalfont\normalsize\bfseries}}
\renewcommand{\subsubsection}{\@startsection{subsubsection}{3}{0mm}%
                                {-1ex plus -.5ex minus -.2ex}%
                                {1ex plus .2ex}%
                                {\normalfont\small\bfseries}}
\makeatother
\setcounter{secnumdepth}{0}
%\setlength{\parindent}{0pt}
%\setlength{\parskip}{0pt plus 0.5ex}

% listings okolje za psevdo kodo
\lstnewenvironment{koda}[1][] %defines the algorithm listing environment
{   
    \lstset{ %this is the stype
        mathescape=true,
        basicstyle=\scriptsize, 
		columns=flexible,
        keywordstyle=\bfseries\em,
        keywords={,vhod, izhod, zacetek, konec, koncamo, ponavljaj, dokler, ce, vrni, za, vsak, vse, v, sicer,} %add the keywords you want, or load a language as Rubens explains in his comment above.
        xleftmargin=.1\textwidth,
		tabsize=4,
		%frame=leftline,xleftmargin=5pt,xrightmargin=5pt,framesep=5pt,
		%inputencoding = utf8,
		extendedchars = true,
		literate={ž}{{\ˇz}}1 {š}{{\ˇs}}1 {č}{{\ˇc}}1 {Ž}{{\ˇZ}}1 {Š}{{\ˇS}}1 {Č}{{\ˇC}}1,
        #1 % this is to add specific settings to an usage of this environment (for instnce, the caption and referable label)
    }
}
{}
% -----------------------------------------------------------------------

\begin{document} 

\begin{multicols}{4}
\setlength{\premulticols}{1pt}
\setlength{\postmulticols}{1pt}
\setlength{\multicolsep}{1pt}
\setlength{\columnsep}{2pt}

\section{Osnove kombinatorike}
\subsection*{Izbori}
Imamo $n$ oštevilčenih kroglic. Na koliko načinov lahko izberemo $k$ kroglic?

\begin{center}
    \begin{tabular}{ m{6em} | c | c | } 
         & \textbf{s pon.} & \textbf{brez pon.}\\ 
        \hline
        \textbf{variacije} \emph{vrstni red je pomemben} & $n^k$ & $n^{\underline{k}}$ \\ 
        \hline
        \textbf{kombinacije} \emph{vrstni red ni pomemben} & $\binom{n+k-1}{k}$ & $\binom{n}{k}$ \\ 
    \end{tabular}
\end{center}

\[\binom{n}{k} = \frac{n^{\underline{k}}}{k!} = \frac{n!}{k!(n-k)!} = \binom{n}{n-k}\]

\section{Elementarna verjetnost}
\begin{align*}
	n \quad & \dots \quad \text{št. ponovitev poskusa} \\
	A \quad & \dots \quad \text{dogodek} \\
	k_n(A) \quad & \dots \quad \text{frekvenca dogodka}
\end{align*}

\textbf{Relativna frekvenca} dogodka $A$:
\[ f_n(A) = \frac{k_n(A)}{n} \]

\subsection{Statistična definicija verjetnosti}
\[ P(A) = \lim_{n \to \infty} f_n(A) \]

\subsection{Klasična definicija verjetnosti}
pri poguju, da so vsi izidi enako verjetni

\[ P(A) = \frac{\text{\# izidov }A}{\text{\# vseh izidov}}\]

\subsection{Geometirjska definicija verjetnosti}
če je število izidov neskončno, pogledamo razmerje ploščine vseh dogodkov in ugodnih dogodkov.

\subsection{Aksiomatična definicija verjetnosti}

Imamo prostor vseh izidov oz. \textbf{vzorčni prostor} $\Omega$. Dogodki so nekatere podmnožice $A \subseteq \Omega$.

\subsubsection{Pravila za računanje z dogodki}
\begin{align*}
	\text{idempotentnost} & \quad A \cup A = A = A \cap A \\
	\text{komutativnost} & \quad A \cup B = B \cup A \\
	& \quad A \cap B = B \cap A \\
	\text{asociativnost} & \quad (A \cup B) \cup C = A \cup ( B \cup C) \\
	& \quad (A \cap B) \cap C = A \cap ( B \cap C) \\
	\text{distibutivnost} & \quad (A \cup B) \cap C = (A \cap C) \cup ( A \cap C) \\
	& \quad (A \cap B) \cup C = (A \cap C) \cup ( A \cap C) \\
	\text{De Morgan} & \quad \big(\bigcup_{i\in I} A_i \big)^\complement = \bigcap_{i \in I} A_i^\complement  \\
	& \quad \big(\bigcap_{i\in I} A_i \big)^\complement = \bigcup_{i \in I} A_i^\complement \\
	& \quad A \cup \Omega = \Omega \quad  A \cap \Omega = A \\
	& \quad A \cup \emptyset = A \quad  A \cap \emptyset = \emptyset \\
	& \quad A \cup A^\complement = \Omega \quad A \cap A^\complement = \emptyset
\end{align*}


Neprazna družina dogodkov $\mathcal{F}$ v $\Omega$ je $\sigma$-algebra, če velja
\begin{itemize}
	\item zaprtost za komplemente: \[ A \in \mathcal{F} \implies A^\complement \in \mathcal{F} \]
	\item zaprtost za števne unije: \[ A_1, A_2, \dots \in \mathcal{F} \implies \bigcup_{i=1}^\infty A_i \in \mathcal{F} \]
\end{itemize}
\textit{Če zahtevamo zaprtost le za končne unije, je $\mathcal{F}$ le algebra.}

Ker je po De Morganovem zakonu $\big(\bigcup_{i\in I} A_i^\complement \big)^\complement = \bigcap_{i \in I} A_i$ imamo zaprtost tudi za preseke.

Ker je $A \setminus B = A \cap B^\complement$ je algebra zaprta tudi za razlike.

Najmanjša algebra je \textbf{trivialna}: $ \{ \emptyset, \Omega \}$.

Največja algebra je: $\mathcal{P}(\Omega)$.

Najmanjša algebra, ki vsebuje $E$ je $\{ \emptyset, E, E^\complement, \Omega \}$.

Dogodka $A$ in $B$ sta \textbf{nezdružljiva} (disjunktna), če je $A \cup B = \emptyset$.

Zaporedje $\{ A_i \}_i \in \mathcal{F}$ (končno ali števno mnogo) je \textbf{popoln sistem dogodkov}, če
\begin{align*}
	\bigcup_i A_i &= \Omega & A_i \cup A_j &= \emptyset, \, \forall i,j: i\neq j
\end{align*}


\textbf{Verjetnost} na $(\Omega, \mathcal{F})$ je preslikava $P: \mathcal{F} \to \mathbb{R}$ z lastnostmi:

\begin{itemize}
	\item $P(A) \geq 0$ za $\forall A \in \mathcal{F}$
	\item $P(\Omega) = 1$
	\item Za paroma nezdružljive dogodke $\{ A_i \}_{i=1}^\infty $ velja \textit{števna aditivnost}
	\[ P(\bigcup_{i=1}^\infty A_i) = \sum_{i=1}^\infty P(A_i)\]
\end{itemize}

Lastnosti $P$:
\begin{itemize}
	\item $P(\emptyset) = 0$
	\item $P$ je končno aditivna.
	\item $P$ je \textit{monotona}: $A \subseteq B \implies P(A) \leq P(B)$
	\item $P(A^\complement) = 1 - P(A)$
	\item $P$ je \textit{zvezna}:
	\[ A_1 \subseteq A_2 \subseteq \dots \implies P\big(\bigcup_{i=1}^\infty\big) = \lim_{i \to \infty} P(A_i)\]
	\[ B_1 \supseteq B_2 \supseteq \dots \implies P\big(\bigcap_{i=1}^\infty\big) = \lim_{i \to \infty} P(B_i)\]
\end{itemize}


\[P(A^\complement) = 1 - P(A)\]
\[ P(A \cup B) = P(A) + P(B) - P(A \cap B) \]

Če lahko prostor izidov razbijemo na paroma nezdružljive enako verjetne dogodke, jih lahko obravnavamo kot izide: če je dogodek $A$ unija $k$ od $n$ takih dogodkov, je $P(A) = k/n$.

\subsection{Načelo vključitev in izključitev}
\begin{align*}
	P(A_1 \cup A_2) &= P(A_1) + P(A_2) - P(A_1 \cap A_2)
\end{align*}
\[ P( \bigcup_{i=1}^n A_i) = \sum_{\emptyset \neq S \subseteq [n]} (-1)^{|S|+1} P(\bigcap_{i\in S} A_i)\]

\section{Verjetnostni prostor}
je trojček $(\Omega, \mathcal{F}, P)$, kjer je $\Omega$ množica vseh izidov, $\mathcal{F}$ $\sigma$-algebra in $P$ preslikava verjetnosti.


Najmanjša algebra $\mathcal{F}$ na $\mathbb{N}$, ki vsebuje $ \{1\}, \{2\}, \dots$, je algebra \[g = \{ A \subseteq \mathbb{N} : \text{$A$ končna ali $A^\complement$ neskončna} \} \]

\subsection{Pogojna verjetnost}
\[ P(A | B) = \frac{P(A \cap B)}{P(B)}\]


\subsection{Izrek o polni verjetnosti}
Če $H_1, H_2, H_3, \dots $ tvorijo \textbf{popoln sistem dogodkov} \textit{(tj. vedno se zgodi natanko eden izmed njih)}, velja:
\[ P(A) = P(H_1) P(A | H_1) + P(H_2) P(A | H_2) + \dots \]
Dogodkom $H_i$ često pravimo hipoteze in jih je lahko končno ali pa števno
neskončno

\subsection{Bayesova formula}
\[ P(H_i | A) = \frac{P(H_i)P(A | H_i)}{P(A)}\]
\[ P(H_i | A) = \frac{P(H_i) P(A | H_i)}{P(H1) P(A | H1) + P(H2) P(A | H2) + \dots } \]

Brezpogojnim verjetnostim $P(H_i)$ pravimo \textbf{apriorne}, pogojnim verjetnostim
$P(H_i | A)$ pa \textbf{aposteriorne} verjetnosti hipotez.

\subsection{Neodvisnot dogodkov}

Dogodka A in B sta \textbf{neodvisna}, če velja:
\[ P(A \cup B) = P(A) P(B) \]
Če je $P(B) > 0$, je to ekvivalentno pogoju, da je $P(A | B) = P(A)$.
Če je $0 < P(B) < 1$, pa je to ekvivalentno tudi pogoju, da je $P(A | B) = P(A | B^\complement)$.
Dogodki $A1, A2, A3, \dots$ so neodvisni, če za poljubne različne indekse $i_1, i_2, \dots , i_k$ velja:
\[ P(A_{i_1} \cap A_{i_2} \cap \dots \cap A_{i_k}) = P(A_{i_1}) P(A_{i_2}) \dots P(A_{i_k}) \]

\subsection{Neodvisnot izpeljanih dogodkov *}
Naj bo $\mathcal{F}$ družina dogodkov. S $\sigma(\mathcal{F})$ označimo najmanjšo $\sigma$-algebro, ki vsebuje
$\mathcal{F}$, tj. družino vseh dogodkov, ki jih dobimo iz dogodkov iz $\mathcal{F}$ s števnimi unijami
in komplementi.
Naj bodo $A_{i,j}$ neodvisni dogodki. Tedaj so tudi poljubni dogodki $B_1 \in \sigma(A_{11}, A_{12}, \dots), B_2 \in \sigma(A_21, A22, . . .), B_3 \in \sigma(A_{31}, A_{32}, \dots), \dots$ neodvisni.

\section{Slučajne spremenljivke}
Slučajna spremenljivka je funkcija $X: \Omega \to \mathbb{R}$ z lastnostijo, da je $\forall x \in \mathbb{R}$ množica $\{ \omega \in \Omega : X(\omega) < x \} \equiv X^{-1}((-\infty, x]) \equiv (X \leq x)$ dogodek.

\subsubsection{Diskretne porazdelitve}
\textbf{Diskretna enakomerna porazdelitev} na $n$ točkah:
\[ X \sim \binom{a_1\ a_2\ \dots\ a_n}{\frac{1}{n}\ \frac{1}{n}\ \dots\ \frac{1}{n}} = \text{Unif} \{ a_1 , \dots , a_n \} \]

\textbf{Binomska porazdelitev} \\
$\text{Bin}(n, p)$, $n \in \mathbb{N}$, $p \in (0,1)$

Naj bo $X$ št. uspelih (z verjetnostjo $p$) poskusov v zaporedju $n$ poskusov. $X \sim \text{Bin}(n, p)$:
\[ p_k = P(X = k) = \binom{n}{k} p^k (1-p)^{n-k}\]

\textbf{Bernulijeva porazdelitev} $Ber(p) \sim Bin(1, p)$

\textbf{Geometrijske porazdelitev} \\
$\text{Geo}(p)$,  $p \in (0,1)$

$(X = k)$ je dogodek, da se $A$ zgodi prvič v $k$-ti ponovitvi.

\[P(X = k) = (1-p)^{k-1} p\]

\textbf{Pascalova / negativna binomska porazdelitev} \\
$Pas(m, p) = NB(m, p)$, $m \in \mathbb{N}$, $p \in (0,1)$

$(X = k)$ je dogodek, da se dogodek $A$ zgodi $m$-tič v $k$-ti ponovitvi.

Oziroma $X$ je število poskusov do vključno $m$-tega uspelega, pri katerig vsak uspe z verjetnostjo $p$. $X \sim Pas(m, p)$:
\[ p_k = P(X = k) = \binom{k-1}{n-1} p^k (1-p)^{k-n}\]

\textbf{Hipergeometrijska porazdelitev} \\
Iz posode, v kateri je $n$ kroglic, od tega $r$ rdečih, na slepo in brez vračanja
izvlečemo $s$ kroglic. Če z $X$ označimo število rdečih med izvlečenimi, ima
ta slučajna spremenljivka hipergeometrijsko porazdelitev: $X \sim \text{Hip}(s, r, n) = \text{Hip}(r, s, n)$. Velja

\[ P(X = k) = \frac{\binom{r}{k} \binom{n-r}{s-k}}{\binom{n}{s}} = \frac{\binom{s}{k} \binom{n-s}{r-k}}{\binom{n}{r}}\]


\subsubsection{Aproksimacija binomske porazdelitve}
\textbf{Poissonova porazdelitev} \\
$Poi(\lambda)$, $\lambda > 0$

\[ p_k = P(X = k) = \frac{\lambda^k}{k!} e^{-\lambda}\]

Če imamo veliko ponovitev ($n \to \infty$) z malo verjetnostjo ($p \to 0$), je $\text{Bin}(n, p) \approx \text{Poi}(np)$

\textbf{Laplaceova lokalna formula: }
Če je $p, 1-p \gg \frac{1}{n}$, lahko $X \sim \text{Bin}(n, p)$ aproksimiramo
\[ P(X = k) \approx \frac{1}{\sigma\sqrt{2\pi}} e^{-\frac{(k-np)^2}{2\sigma^2}} \quad \sigma = \sqrt{np(1-p)} \]

\textbf{Laplaceova integralska formula: }
\[ P(a \leq X \leq b) \approx \Phi\left( \frac{b - np}{\sigma} \right) - \Phi\left( \frac{a - np}{\sigma} \right) \]

Za majhno relativno napako zahtevamo še: 
\begin{itemize}
	\item $|a-np| \ll \sigma^{4/3}$ ali $|b-np| \ll \sigma^{4/3}$ 
	\item $a, b \in \mathbb{Z} + \frac{1}{2}$ ali $b - a \gg 1$
\end{itemize}

\subsubsection{Kumulativna porazdelitvena funkcija}
\[ F_X(x) = P(X \leq x)\]

\subsection{Zvezno porazdeljene slučajne spremenljivke}
Realna slučajna spremenljivka $X$ je porazdeljena zvezno, če obstaja taka integrabilna funkcija $p_X: \mathbb{R} \to [0, \infty)$, da za poljubna $a \leq b$ velja:
\[ P( a \leq X \leq b) = \int_a^b p_X(t) dx \]
Funkciji $p_X$ pravimo \textbf{porazdelitvena gostota}


Komulativna funkcija slučajne spremenljivke $X$
\[ F_X(x) = \int_{\infty}^x p_X(t) dt \]

Če je $F_X$ zvezna in odsekoma zvezno odvedljiva, je $X$ porazdeljena zvezno in za vse razen za končno mnogo točk $x$ velja $p_x(x) = F'_X(x)$.

\subsection{Zvezne porazdelitve}
\textbf{Enakomerna zvezna porazdelitev} na $ [a, b] $

\[ p(x) = \begin{cases}
	\frac{1}{b-a} & a \leq x \leq b \\
	0 & \text{sicer}
\end{cases}\]

\[ F(x) = \begin{cases}
	0 & x \leq a \\
	\frac{x-a}{b-a} & a \leq x \leq b \\
	1 & b \leq x
\end{cases}\]

\textbf{Normalna / Gaussova porazdelitev} $N(\mu, \sigma)$
\[ p(x) = \frac{1}{\sigma \sqrt{2\pi}} e^{-\frac{1}{2} \left( \frac{x-\mu}{\sigma} \right)^2}\]

\[ F(x) = \frac{1}{2} + \Phi\left(\frac{x-\mu}{\sigma}\right)\]

Standardizirana normalna porazdelitev: $N(0, 1)$

\textbf{Eksponentna porazdelitev} $\text{Exp}(\lambda)$
\[ p(x) = \begin{cases}
	\lambda e^{-\lambda x} & x \geq 0 \\
	0 & x < 0
\end{cases}\]

\[ F(x) = \begin{cases}
	1-e^{-\lambda x} & x \geq 0 \\
	0 & x < 0
\end{cases}\]


\textbf{Porazdelitev} $\Gamma(b,c)$, $b > 0$, $c > 0$
\[ p(x) = \begin{cases}
	\frac{c^b}{\Gamma(b)} x^{b-1} e^{-cx} & x \geq 0 \\
	0 & x < 0
\end{cases}\]

\textbf{Porazdelitev} $\chi^2(n)$ \\
$n \in \mathbb{N}$ je št. prostorskih stopenj

\[ \chi^2(n) = \Gamma(\frac{n}{2}), \frac{1}{2})\]

\textbf{Cauchyjeva porazdelitev}
\[ p(x) = \frac{1}{\pi (1+x^2)}\]
\[ F(x) = \frac{1}{\pi} \arctan x + \frac{1}{2}\]

\section{Slučajni vektorji}
Slučajni vektor je $n$-terica slučajnih spremenljivk $X = (X_1, \dots , X_n): \Omega -> R$ 

\subsubsection{Porazdelitvena funkcija slučajnega vektorja}
\[ F_{(X_1, \dots , X_n)}(x_1, \dots , x_n) = P(X_1 \leq x_1, \dots , X_n \leq x_n)\]

\subsection{Neodvisnot slučajnih spremenljivk}
Slučajne spremenljivke so neodvisne, če je
\[ F_{(X_1, \dots , X_n)}(x_1, \dots, x_n) = F_{X_1}(x_1) \dots F_{X_n}(x_n)\]
Torej so dogodki $(X_1 \leq x_1), \dots , (X_n \leq x_n)$ neodvisni.

Naj bo $(X, Y)$ diskretno porazdeljen sluč. vektor:
\[ p_{ij} = P(X \leq x_i, Y \leq y_j) \quad p_i = P(X \leq x_i) \quad P(Y \leq y_j) \]
potem velja:
\[ X, Y \text{neodvisni} \iff p_{ij} = p_i q_j \]


Naj bo $(X, Y)$ zvezno porazdeljen sluč. vektor z gostoto $ p_{(X, Y)}(x,y)$, potem velja:
\[ X, Y \text{neodvisni} \iff \exists p_X, p_Y: p_{(X,Y)}(x,y) = p_X(x) p_Y(y)\]

\subsection{Funkcije slučajnih spremenljivk}
Naj bosta $A, B \subseteq \mathbb{R}$ odprti množici in $h: A \to B$ taka bijekcija, da je funkcija
$h^{-1}: B \to A$ zvezno odvedljiva. Nadalje naj bo $X$ zvezno porazdeljena slučajna
spremenljivka z gostoto $p_X$, ki je izven množice A enaka nič. Tedaj je slučajna
spremenljivka $Y := h(X)$ porazdeljena zvezno z gostoto:

\[ p_Y(y) = \begin{cases}
	p_X\left( h^{-1}(y) \right) \left|(h^{-1})'(y)\right| & y \in B \\
	0 & sicer	
\end{cases}\]

Naj bo $X$ zvezno porazdeljena slučajna spremenljivka z gostoto $p_X$, skoncentrirana na odprti množici $A$. Naj bo $h: A \to \mathbb{R}$ zvezno odvedljiva in $h'(x) \neq 0$ za $\forall x \in A$. 
Tedaj je slučajna spremenljivka $Y = h(X)$ porazdeljena zvezno z gostoto:

\[ p_Y = \sum_{\substack{x \in A \\ h(x) = y}} \frac{p_X(x)}{|h'(x)|}\]


\end{multicols}

\end{document}