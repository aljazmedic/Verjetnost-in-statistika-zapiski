\documentclass[a4paper,8pt]{extarticle}
\usepackage[utf8]{inputenc}

\usepackage{fancyhdr}

\usepackage[pdftex]{graphicx} % Required for including pictures
\usepackage[pdftex,linkcolor=black,pdfborder={0 0 0}]{hyperref} % Format links for pdf
\usepackage{calc} % To reset the counter in the document after title page
\usepackage{enumitem} % Includes lists

\usepackage{textcomp}
\usepackage{eurosym}

\usepackage{ dsfont } % font za množice
% tabele
\usepackage{array}
\usepackage{wrapfig}

\usepackage{tikz,forest}
\usetikzlibrary{arrows.meta}

\frenchspacing % No double spacing between sentences
\setlength{\parindent}{0pt}
\setlength{\parskip}{0.7em}

\usepackage{mathtools}
\usepackage{blkarray, bigstrut} %


\usepackage{amssymb,amsmath,amsthm,amsfonts}
\usepackage{multicol,multirow}
\usepackage{calc}
\usepackage{ifthen}
\usepackage{tabularx}
\usepackage[landscape]{geometry}
\usepackage{listings}
\usepackage{inconsolata}
%\usepackage[colorlinks=true,citecolor=blue,linkcolor=blue]{hyperref}
%\usepackage{accents}

\newcommand{\vect}[1]{\accentset{\rightharpoonup}{#1}}

\ifthenelse{\lengthtest { \paperwidth = 11in}}
    { \geometry{top=.5in,left=.5in,right=.5in,bottom=.5in} }
	{\ifthenelse{ \lengthtest{ \paperwidth = 297mm}}
		{\geometry{top=1cm,left=1cm,right=1cm,bottom=1cm} }
		{\geometry{top=1cm,left=1cm,right=1cm,bottom=1cm} }
	}
\pagestyle{empty}
\makeatletter
\renewcommand{\section}{\@startsection{section}{1}{0mm}%
                                {-1ex plus -.5ex minus -.2ex}%
                                {0.5ex plus .2ex}%x
                                {\normalfont\large\bfseries}}
\renewcommand{\subsection}{\@startsection{subsection}{2}{0mm}%
                                {-1explus -.5ex minus -.2ex}%
                                {0.5ex plus .2ex}%
                                {\normalfont\normalsize\bfseries}}
\renewcommand{\subsubsection}{\@startsection{subsubsection}{3}{0mm}%
                                {-1ex plus -.5ex minus -.2ex}%
                                {1ex plus .2ex}%
                                {\normalfont\small\bfseries}}
\makeatother
\setcounter{secnumdepth}{0}
%\setlength{\parindent}{0pt}
%\setlength{\parskip}{0pt plus 0.5ex}

% listings okolje za psevdo kodo
\lstnewenvironment{koda}[1][] %defines the algorithm listing environment
{   
    \lstset{ %this is the stype
        mathescape=true,
        basicstyle=\scriptsize, 
		columns=flexible,
        keywordstyle=\bfseries\em,
        keywords={,vhod, izhod, zacetek, konec, koncamo, ponavljaj, dokler, ce, vrni, za, vsak, vse, v, sicer,} %add the keywords you want, or load a language as Rubens explains in his comment above.
        xleftmargin=.1\textwidth,
		tabsize=4,
		%frame=leftline,xleftmargin=5pt,xrightmargin=5pt,framesep=5pt,
		%inputencoding = utf8,
		extendedchars = true,
		literate={ž}{{\ˇz}}1 {š}{{\ˇs}}1 {č}{{\ˇc}}1 {Ž}{{\ˇZ}}1 {Š}{{\ˇS}}1 {Č}{{\ˇC}}1,
        #1 % this is to add specific settings to an usage of this environment (for instnce, the caption and referable label)
    }
}
{}
% -----------------------------------------------------------------------

\begin{document} 

\begin{multicols}{4}
\setlength{\premulticols}{1pt}
\setlength{\postmulticols}{1pt}
\setlength{\multicolsep}{1pt}
\setlength{\columnsep}{2pt}


\section{Verjetnost}
\begin{align*}
	n \quad & \dots \quad \text{št. ponovitev poskusa} \\
	A \quad & \dots \quad \text{dogodek} \\
	k_n(A) \quad & \dots \quad \text{frekvenca dogodka}
\end{align*}

\textbf{Relativna frekvenca} dogodka $A$:
\[ f_n(A) = \frac{k_n(A)}{n} \]

\subsection{Statistična definicija verjetnosti}
\[ P(A) = \lim_{n \to \infty} f_n(A) \]

\subsection{Klasična definicija verjetnosti}
pri poguju, da so vsi izidi enako verjetni

\[ P(A) = \frac{\text{\# izidov }A}{\text{\# vseh možnih izidov}}\]

\subsection{Geometirjska definicija verjetnosti}
če je število izidov neskončno, pogledamo razmerje ploščine vseh dogodkov in ugodnih dogodkov.

\subsection{Aksiomatična definicija verjetnosti}

Imamo prostor vseh izidov oz. \textbf{vzorčni prostor} $\Omega$. Dogodki so nekatere podmnožice $A \subseteq \Omega$.

\subsubsection{Pravila za računanje z dogodki}
\begin{align*}
	\text{idempotentnost} & \quad A \cup A = A = A \cap A \\
	\text{komutativnost} & \quad A \cup B = B \cup A \\
	& \quad A \cap B = B \cap A \\
	\text{asociativnost} & \quad (A \cup B) \cup C = A \cup ( B \cup C) \\
	& \quad (A \cap B) \cap C = A \cap ( B \cap C) \\
	\text{distibutivnost} & \quad (A \cup B) \cap C = (A \cap C) \cup ( A \cap C) \\
	& \quad (A \cap B) \cup C = (A \cap C) \cup ( A \cap C) \\
	\text{De Morgan} & \quad \big(\bigcup_{i\in I} A_i \big)^\complement = \bigcap_{i \in I} A_i^\complement  \\
	& \quad \big(\bigcap_{i\in I} A_i \big)^\complement = \bigcup_{i \in I} A_i^\complement
\end{align*}

Neprazna družina dogodkov $\mathcal{F}$ v $\Omega$ je $\sigma$-algebra, če velja
\begin{itemize}
	\item zaprtost za komplemente: \[ A \in \mathcal{F} \implies A^\complement \in \mathcal{F} \]
	\item zaprtost za števne unije: \[ A_1, A_2, \dots \in \mathcal{F} \implies \bigcup_{i=1}^\infty A_i \in \mathcal{F} \]
\end{itemize}
\textit{Če zahtevamo zaprtost le za končne unije, je $\mathcal{F}$ le algebra.}

Ker je po De Morganovem zakonu $\big(\bigcup_{i\in I} A_i^\complement \big)^\complement = \bigcap_{i \in I} A_i$ imamo zaprtost tudi za preseke.

Ker je $A \setminus B = A \cap B^\complement$ je algebra zaprta tudi za razlike.

Najmanjša algebra je \textbf{trivialna}: $ \{ \emptyset, \Omega \}$.

Največja algebra je: $\mathcal{P}(\Omega)$.

Najmanjša algebra, ki vsebuje $E$ je $\{ \emptyset, E, E^\complement, \Omega \}$.

Dogodka $A$ in $B$ sta \textbf{nezdružljiva} (disjunktna), če je $A \cup B = \emptyset$.

Zaporedje $\{ A_i \}_i \in \mathcal{F}$ (končno ali števno mnogo) je \textbf{popoln sistem dogodkov}, če
\begin{align*}
	\bigcup_i A_i &= \Omega & A_i \cup A_j &= \emptyset, \, \forall i,j: i\neq j
\end{align*}


\textbf{Verjetnost} na $(\Omega, \mathcal{F})$ je preslikava $P: \mathcal{F} \to \mathbb{R}$ z lastnostmi:

\begin{itemize}
	\item $P(A) \geq 0$ za $\forall A \in \mathcal{F}$
	\item $P(\Omega) = 1$
	\item Za paroma nezdružljive dogodke $\{ A_i \}_{i=1}^\infty $ velja \textit{števna aditivnost}
	\[ P(\bigcup_{i=1}^\infty A_i) = \sum_{i=1}^\infty P(A_i)\]
\end{itemize}

Lastnosti $P$:
\begin{itemize}
	\item $P(\emptyset) = 0$
	\item $P$ je končno aditivna.
	\item $P$ je \textit{monotona}: $A \subseteq B \implies P(A) \leq P(B)$
	\item $P(A^\complement) = 1 - P(A)$
	\item $P$ je \textit{zvezna}:
	\[ A_1 \subseteq A_2 \subseteq \dots \implies P\big(\bigcup_{i=1}^\infty\big) = \lim_{i \to \infty} P(A_i)\]
	\[ B_1 \supseteq B_2 \supseteq \dots \implies P\big(\bigcap_{i=1}^\infty\big) = \lim_{i \to \infty} P(B_i)\]
\end{itemize}

\section{Verjetnostni prostor}
je trojček $(\Omega, \mathcal{F}, P)$, kjer je $\Omega$ množica vseh izidov, $\mathcal{F}$ $\sigma$-algebra in $P$ preslikava verjetnosti.


Najmanjša algebra $\mathcal{F}$ na $\mathbb{N}$, ki vsebuje $ \{1\}, \{2\}, \dots$, je algebra \[g = \{ A \subseteq \mathbb{N} : \text{$A$ končna ali $A^\complement$ neskončna} \} \]

\section{Pogojna verjetnost}
\[ P(A | B) = \frac{P(A \cap B)}{P(B)}\]


\end{multicols}

\end{document}